% intro.tex

%%%%%%%%%%%%%%%%%%%%
\begin{frame}{Overview of the Work on \unistore}
  \begin{center}
    \fig{width = 0.80\textwidth}{figs/unistore-atc21}
    \vspace{0.20cm}
    \red{ATC'2021 (CCF A)}

    \vspace{1.00cm}
    \unistore{} is the first \red{fault-tolerant} and scalable \cyan{transactional} data store
    that \blue{combines causal and strong consistency}.
  \end{center}
\end{frame}
%%%%%%%%%%%%%%%%%%%%

%%%%%%%%%%%%%%%%%%%%
\begin{frame}{Overview of the Work on \unistore}
  \begin{center}
    {\red{\underline{P}}artial \red{\underline{O}}rder-\red{\underline{R}}estrictions (PoR) Consistency}

    \vspace{0.30cm}
    CC $<$ PoR $<$ SER \\[6pt]
    CC: \cc; SER: \ser

    \vspace{1.20cm}
    \begin{columns}
      \column{0.05\textwidth}
      \column{0.90\textwidth}
      \begin{description}[Key Challenges (II):]
        \item[\red{Key Challenges (I):}] Ensure liveness in presence of faults \\[10pt]
        \item[\red{Key Challenges (II):}] Provide rigorous correctness proof
      \end{description}
      \column{0.05\textwidth}
    \end{columns}
  \end{center}
\end{frame}
%%%%%%%%%%%%%%%%%%%%

%%%%%%%%%%%%%%%%%%%%
\begin{frame}{Overview of the Work on \unistore}
  \begin{center}
    \fig{width = 0.80\textwidth}{figs/unistore-atc21}
    \vspace{0.20cm}
    \red{ATC'2021 (CCF A)}

    \vspace{1.00cm}
    \noindent I am fully responsible for the rigorous correctness proof: \\[3pt]

    \begin{itemize}
      \item Finished a proof of 20 pages contained in the \textsf{arXiv} version
      \item Identified several nontrivial bugs
            in the early versions of the protocol\footnote{
              One of these bugs also exists in the well-known Granola protocol
              proposed by James Cowling and Barbara Liskov,
              something that had gone unnoticed for 10 years.}
    \end{itemize}
  \end{center}
\end{frame}
%%%%%%%%%%%%%%%%%%%%

%%%%%%%%%%%%%%%%%%%%
\begin{frame}{What is \unistore?}
  \begin{center}
    \unistore{} is a \red{fast}, \red{scalable}, and \red{fault-tolerant} \\[8pt]
    \cyan{transactional} distributed key-value store \\[8pt]
    that supports a \blue{combination of weak and strong consistency}.

    \pause
    \vspace{0.80cm}
    Weak consistency: \cc \\[6pt]
    Strong consistency: \ser
  \end{center}
\end{frame}
%%%%%%%%%%%%%%%%%%%%

%%%%%%%%%%%%%%%%%%%%
\begin{frame}{Why \textsc{Uni-}?}
  \begin{center}
    Weak consistency: low latency, high availability, \\[3pt]
    but unable to preserve critical application invariants

    \fig{width = 0.40\textwidth}{figs/best-of-both-worlds}

    Strong consistency: easy to preserve critical application invariants, \\[3pt]
    but require global synchronization
  \end{center}
\end{frame}
%%%%%%%%%%%%%%%%%%%%

%%%%%%%%%%%%%%%%%%%%
\begin{frame}{The \textsc{Uni-} Approach}
  \begin{itemize}
    \setlength{\itemsep}{8pt}
    \item Allow \red{multiple consistency levels} to coexist
    \item Take the \blue{weak consistency level} as the default and the baseline
    \item Allow programmers to \red{choose} the operations that should be executed under \blue{strong consistency}
      \begin{itemize}
        \item e.g., if the execution of an operation may violate the application violations
      \end{itemize}
  \end{itemize}
\end{frame}
%%%%%%%%%%%%%%%%%%%%

%%%%%%%%%%%%%%%%%%%%
\begin{frame}{The \textsc{Uni-} Approach}
  \begin{center}
    {\red{\underline{P}}artial \red{\underline{O}}rder-\red{\underline{R}}estrictions (PoR) Consistency} \\[3pt]
    \citepf{Li@OSDI'2012, Li@ACT'2018}
  \end{center}

  \begin{itemize}
    \setlength{\itemsep}{8pt}
    \item PoR allows programmers to classify operations as either \blue{causal} or \red{strong}.
    \pause
    \item \blue{Causal} operations satisfy \cc:
      \begin{itemize}
        \item Clients see updates in an order that respects the \cyan{potential causality} between them. \\[3pt]
        \item \cyan{Causally independent} operations can be executed concurrently.
      \end{itemize}
    \pause
    \item Programmers use \red{strong} operations to enforce orders
      between some causally independent operations.
  \end{itemize}
\end{frame}
%%%%%%%%%%%%%%%%%%%%

%%%%%%%%%%%%%%%%%%%%
\begin{frame}{\ser{} and \cc}
  \begin{center}
    TODO: +fig
    Clients see updates in an order that respects the \red{potential causality} between them. \\[3pt]
    Causally independent operations can be executed concurrently.
  \end{center}
\end{frame}
%%%%%%%%%%%%%%%%%%%%

%%%%%%%%%%%%%%%%%%%%
\begin{frame}{\ser{} and \cc}
  \begin{center}
    \fig{width = 0.60\textwidth}{figs/serializability-jepsen}
    All the operations seem to be executed in some \red{sequential} order.
  \end{center}
\end{frame}
%%%%%%%%%%%%%%%%%%%%

%%%%%%%%%%%%%%%%%%%%
\begin{frame}{A Banking Application}
  \begin{center}
    \textsc{deposit \quad withdraw \quad query \quad interest}
    \fig{width = 0.60\textwidth}{figs/alice-bob}
    \red{Invariant: $\textsf{balance} \ge 0$}

    \pause
    \vspace{0.50cm}
    \blue{\textsc{deposit}} operations can be executed under \blue{\cc}.

    \pause
    \vspace{0.50cm}
    {\blue{\textsc{deposit($50$)}}}
  \end{center}
\end{frame}
%%%%%%%%%%%%%%%%%%%%

%%%%%%%%%%%%%%%%%%%%
\begin{frame}{A Banking Application}
  \begin{center}
    \textsc{deposit \quad withdraw \quad query \quad interest}
    \fig{width = 0.60\textwidth}{figs/alice-bob}
    \red{Invariant: $\textsf{balance} \ge 0$}

    \pause
    \vspace{0.50cm}
    {Causal consistency also allows two causally independent \red{\textsc{withdraw}} to execute concurrently,
    without knowing each other.}

    \pause
    \vspace{0.50cm}
    {\blue{\textsc{withdraw($60$)}}}
  \end{center}
\end{frame}
%%%%%%%%%%%%%%%%%%%%

%%%%%%%%%%%%%%%%%%%%
\begin{frame}{The PoR Approach}
  % \begin{definition}[Session Order]
  %   A transaction $t_{1}$ precedes a transaction $t_{2}$
  %   in the \red{session order}, denoted \red{$t_{1} \rel{\so} t_{2}$},
  %   if they are executed by the same client and $t_{1}$ is executed before $t_{2}$.
  % \end{definition}

  \begin{itemize}
    \setlength{\itemsep}{8pt}
    \item The programmer provides a symmetric \red{conflict relation}
      $\conflict$ on operations.
    \item Any operation involved in the conflict relation is marked \blue{strong}.
    \item PoR ensures that conflicting operations are executed serially.
  \end{itemize}
\end{frame}
%%%%%%%%%%%%%%%%%%%%

%%%%%%%%%%%%%%%%%%%%
\begin{frame}{A Banking Application}
  \begin{center}
    \textsc{deposit \quad withdraw \quad query \quad interest}
    \fig{width = 0.60\textwidth}{figs/alice-bob}
    \red{Invariant: $\textsf{balance} \ge 0$}

    \vspace{0.50cm}
    Only \red{\textsc{withdraw}} are marked \red{strong}.

    Declaring that strong transactions \\[3pt]
    including \textsc{withdraw} on the same account conflict.
  \end{center}
\end{frame}
%%%%%%%%%%%%%%%%%%%%

%%%%%%%%%%%%%%%%%%%%
\begin{frame}{Consistency Model of \unistore (Safety)}
  \begin{center}
    \unistore{} implements a \blue{transactional} variant of \\
    the \por{} consistency

    \pause
    \[
      \txs \triangleq \causaltxs \uplus \strongtxs
    \]

    \pause
    \vspace{0.60cm}
    \begin{enumerate}[(I)]
      \centering
      \item transactional causal consistency by default
      \item to specify conflicting transactions under strong consistency
    \end{enumerate}
  \end{center}
\end{frame}
%%%%%%%%%%%%%%%%%%%%

%%%%%%%%%%%%%%%%%%%%
\begin{frame}{Consistency Model of \unistore (Liveness)}
  \ev

  \vspace{0.50cm}
  A transaction $t \in T$ that is either \purple{strong}
  or \teal{originates at a correct data center}
  eventually become \red{visible} at all \violet{correct} data centers.
  \begin{itemize}
    \item from some point on, $t$ precedes all transactions issued
      at correct data centers.
  \end{itemize}
\end{frame}
%%%%%%%%%%%%%%%%%%%%

%%%%%%%%%%%%%%%%%%%%
\begin{frame}{Design Challenges of \unistore}
  \begin{center}
    {How to satisfy \red{liveness (\ev)} despite failures?}
    \fig{width = 0.50\textwidth}{figs/design-challenges}

    \vspace{0.50cm}
    A transaction $t \in T$ that is either \purple{strong}
    or \teal{originates at a correct data center}
    eventually become \red{visible} at all \violet{correct} data centers.
  \end{center}
\end{frame}
%%%%%%%%%%%%%%%%%%%%

%%%%%%%%%%%%%%%%%%%%
\begin{frame}{Liveness of Causal Transactions}
  \begin{center}
    {Data center $d_{1}$ crashes \\[3pt]
      before $\causalcolor{t_{1}}$ is replicated to correct data center $d_{3}$.}

    \vspace{0.20cm}
    \fig{width = 0.70\textwidth}{figs/liveness-causal}

    \pause
    Transaction $t_{2}$ (at correct data center $d_{2}$) \\[3pt]
    may never become visible at correct data center $d_{3}$.
  \end{center}
\end{frame}
%%%%%%%%%%%%%%%%%%%%

%%%%%%%%%%%%%%%%%%%%
\begin{frame}{Liveness of Causal Transactions}
  \begin{center}
    Data center $d_{1}$ crashes \\[3pt]
    before $\causalcolor{t_{1}}$ is replicated to correct data center $d_{3}$.

    \vspace{0.20cm}
    \fig{width = 0.70\textwidth}{figs/liveness-causal}

    \pause
    Data center $d_{2}$ need to \red{forward} causal transactions \\[3pt]
    to other data centers.
  \end{center}
\end{frame}
%%%%%%%%%%%%%%%%%%%%

%%%%%%%%%%%%%%%%%%%%
\begin{frame}{Liveness of Strong Transactions}
  \begin{center}
    Data center $d_{1}$ crashes \\[3pt]
    before $\causalcolor{t_{1}}$ is replicated to correct data center $d_{3}$.

    \vspace{0.50cm}
    \fig{width = 1.00\textwidth}{figs/liveness-strong}

    \pause
    \vspace{0.30cm}
    Transaction $\strongcolor{t_{2}}$ will never be visible at $d_{3}$. \\[3pt]
    No transaction $\strongcolor{t_{3}}$ conflicting with $\strongcolor{t_{2}}$ can commit.
  \end{center}
\end{frame}
%%%%%%%%%%%%%%%%%%%%

%%%%%%%%%%%%%%%%%%%%
\begin{frame}{Liveness of Strong Transactions}
  \begin{center}
    \unistore{} ensures that before a strong transaction commits, \\[3pt]
    all its causal dependencies are \red{uniform}, \\[3pt]
    i.e., will eventually become visible at all correct data centers.

    \vspace{0.30cm}
    \fig{width = 1.00\textwidth}{figs/liveness-strong}

    \pause
    Transaction $\causalcolor{t_{1}}$ will eventually be visible at $d_{3}$. \\[3pt]
    Transaction $\strongcolor{t_{2}}$ will eventually be visible at $d_{3}$. \\[3pt]
    Transaction $\strongcolor{t_{3}}$ may be committed at $d_{3}$. \\[3pt]
  \end{center}
\end{frame}
%%%%%%%%%%%%%%%%%%%%

%%%%%%%%%%%%%%%%%%%%
\begin{frame}{Performance of \unistore}
  \begin{center}
    Causal transactions remain highly-available, i.e., committed locally. \\[6pt]

    \vspace{0.30cm}
    \fig{width = 1.00\textwidth}{figs/liveness-strong}

    \vspace{0.20cm}
    A strong transaction may have to \red{wait} for some of its dependencies
    to become uniform before committing.

    \pause
    \vspace{0.50cm}
    \red{However, this may cost too much.}
  \end{center}
\end{frame}
%%%%%%%%%%%%%%%%%%%%

%%%%%%%%%%%%%%%%%%%%
\begin{frame}{Performance of \unistore{}}
  \begin{center}
    \unistore{} makes a remote causal transaction visible to clients \\[3pt]
    only \red{after it is uniform}.

    \vspace{0.30cm}
    Causal transactions are executed on an (almost) \\[3pt]
    \red{uniform snapshot} that may be slightly in the past.

    \vspace{0.50cm}
    \fig{width = 1.00\textwidth}{figs/liveness-strong}

    \pause
    A strong transaction only needs to wait for causal transactions \\[3pt]
    originating at the \red{local} data center to become uniform.
  \end{center}
\end{frame}
%%%%%%%%%%%%%%%%%%%%

%%%%%%%%%%%%%%%%%%%%
\begin{frame}{Scalability of \unistore}
  \begin{center}
    \unistore{} scales horizontally, \\[3pt]
    i.e., with the number of machines ({partitions}) in each data center.
  \end{center}
\end{frame}
%%%%%%%%%%%%%%%%%%%%