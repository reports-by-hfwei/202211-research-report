% intro.tex

%%%%%%%%%%%%%%%%%%%%
\begin{frame}{Overview of the Work on \unistore}
  \begin{center}
    \fig{width = 0.80\textwidth}{figs/unistore-atc21}
    \vspace{0.20cm}
    \red{USENIX ATC'2021}

    \vspace{1.00cm}
    \unistore{} is the first \red{fault-tolerant} and scalable \cyan{transactional} data store
    that \blue{combines weak and strong consistency}.
  \end{center}
\end{frame}
%%%%%%%%%%%%%%%%%%%%

%%%%%%%%%%%%%%%%%%%%
\begin{frame}{Overview of the Work on \unistore}
  \begin{center}
    % {\red{\underline{P}}artial \red{\underline{O}}rder-\red{\underline{R}}estrictions (PoR) Consistency}

    % \vspace{0.30cm}
    % CC $<$ PoR $<$ SER \\[6pt]
    % CC: \cc; SER: \ser

    % \pause
    % \vspace{1.00cm}
    \begin{columns}
      \column{0.05\textwidth}
      \column{0.90\textwidth}
      \begin{description}[Challenges (II):]
        \setlength{\itemsep}{15pt}
        \item[Challenges (I):] To ensure \red{liveness} of the system in the presence of data
          center failures \pause
        \item[Challenges (II):] To provide a rigorous correctness \red{proof} of the protocol
          which is rather complicated
      \end{description}
      \column{0.05\textwidth}
    \end{columns}
  \end{center}
\end{frame}
%%%%%%%%%%%%%%%%%%%%

%%%%%%%%%%%%%%%%%%%%
\begin{frame}{Overview of the Work on \unistore}
  \begin{center}
    \fig{width = 0.80\textwidth}{figs/unistore-atc21}
    \vspace{0.20cm}
    \red{ATC'2021 (CCF A)}

    \pause
    \vspace{0.80cm}
    \noindent I am fully responsible for developing a rigorous correctness proof: \\[3pt]

    \pause
    \begin{itemize}[<+->]
      \item Finished a proof of 20 pages contained in the \textsf{arXiv} version
      \item Identified several nontrivial bugs in the early versions of the
            protocol\footnote{ One of these bugs also exists in the well-known Granola
              protocol proposed by James Cowling and Barbara Liskov, something that had gone
              unnoticed for 10 years.}, and proposed solutions to fix them
    \end{itemize}
  \end{center}
\end{frame}
%%%%%%%%%%%%%%%%%%%%

%%%%%%%%%%%%%%%%%%%%
\begin{frame}{What is \unistore?}
  \begin{center}
    \unistore{} is a fast, scalable, and \red{fault-tolerant} \\[8pt]
    transactional distributed key-value store \\[8pt]
    that supports a \red{combination of weak and strong consistency}.
  \end{center}
\end{frame}
%%%%%%%%%%%%%%%%%%%%

%%%%%%%%%%%%%%%%%%%%
\begin{frame}{Why \textsc{Uni-}?}
  \begin{center}
    \causalcolor{Weak Consistency:}
    \uncover<2->{low latency, high availability, \\[3pt]
      but hard to preserve application invariants}

    \uncover<4->{\fig{width = 0.40\textwidth}{figs/best-of-both-worlds}}

    \strongcolor{Strong Consistency:}
    \uncover<3->{easy to preserve application invariants, \\[3pt]
      but require global synchronization among data centers}
  \end{center}
\end{frame}
%%%%%%%%%%%%%%%%%%%%

%%%%%%%%%%%%%%%%%%%%
% \begin{frame}{The \textsc{Uni-} Approach}
%   \begin{itemize}[<+->]
%     \setlength{\itemsep}{8pt}
%     \item Multiple consistency levels coexist in a store
%     \item Take \causalcolor{weak consistency} as the default
%     \item Programmers can choose the transactions that should be executed under
%           \strongcolor{strong consistency} \\[3pt]
%           \begin{itemize}
%             \item e.g., if the execution of a set of transactions may violate the application
%                   invariants
%           \end{itemize}
%   \end{itemize}
% \end{frame}
%%%%%%%%%%%%%%%%%%%%

%%%%%%%%%%%%%%%%%%%%
\begin{frame}{The \textsc{Uni-} Approach}
  \begin{center}
    {\red{\underline{P}}artial \red{\underline{O}}rder-\red{\underline{R}}estrictions (PoR) Consistency} \\[3pt]
    \citepf{Li@OSDI'2012, Li@ACT'2018}
  \end{center}

  \pause
  \begin{itemize}
    \setlength{\itemsep}{8pt}
    \item PoR allows programmers to classify transactions as either \causalcolor{causal}
          or \strongcolor{strong}. \pause
    \item \causalcolor{Causal} transactions satisfy \cc:
          \begin{itemize}
            \item Clients see updates in an order that respects the \cyan{potential causality}
                  between them. \\[3pt]
            \item \cyan{Causally independent} transactions can be executed concurrently.
          \end{itemize}
          \pause
    \item Programmers use \strongcolor{strong} transactions to enforce orders between
          some conflicting causally independent transactions.
          \begin{itemize}
            \item Either $t$ is visible to $t'$ or $t'$ is visible to $t$.
          \end{itemize}
  \end{itemize}
\end{frame}
%%%%%%%%%%%%%%%%%%%%

%%%%%%%%%%%%%%%%%%%%
% \begin{frame}{\cc}
%   \fig{width = 0.50\textwidth}{figs/cc-illustration}
% \end{frame}
%%%%%%%%%%%%%%%%%%%%

%%%%%%%%%%%%%%%%%%%%
% \begin{frame}{\ser}
%   \begin{center}
%     \fig{width = 0.60\textwidth}{figs/serializability-jepsen}
%     All transactions seem to be executed in some \red{sequential} order.
%   \end{center}
% \end{frame}
%%%%%%%%%%%%%%%%%%%%

%%%%%%%%%%%%%%%%%%%%
\begin{frame}{A Banking Application}
  \begin{center}
    % \textsc{deposit \quad withdraw \quad query \quad interest}
    \fig{width = 0.60\textwidth}{figs/alice-bob}
    \red{Invariant: $\textsf{balance} \ge 0$}

    \pause
    \vspace{0.50cm}
    \causalcolor{\textsc{deposit}} operations can be executed under \causalcolor{\cc}: \\[3pt]
    locally commit first, then synchronize the updates asynchronously

    \pause
    \vspace{0.30cm}
    {\causalcolor{\textsc{deposit($50$)}}}
  \end{center}
\end{frame}
%%%%%%%%%%%%%%%%%%%%

%%%%%%%%%%%%%%%%%%%%
\begin{frame}{A Banking Application}
  \begin{center}
    % \textsc{deposit \quad withdraw \quad query \quad interest}
    \fig{width = 0.60\textwidth}{figs/alice-bob}
    \red{Invariant: $\textsf{balance} \ge 0$}

    \pause
    \vspace{0.50cm}
    {However, \cc{} also allows two causally independent \strongcolor{\textsc{withdraw}} to execute concurrently,
      without knowing each other.}

    \pause
    \vspace{0.30cm}
    {\causalcolor{\textsc{withdraw($60$)}}} produces $-20 < 0$
  \end{center}
\end{frame}
%%%%%%%%%%%%%%%%%%%%

%%%%%%%%%%%%%%%%%%%%
% \begin{frame}{The PoR Approach}
%   \begin{itemize}[<+->]
%     \setlength{\itemsep}{8pt}
%     \item The programmer provides a symmetric \strongcolor{conflict relation} $\conflict$
%           on transactions.
%     \item Any transaction involved in the conflict relation is marked
%           \strongcolor{strong}.
%     \item PoR ensures that conflicting transactions are executed serially.
%   \end{itemize}
% \end{frame}
%%%%%%%%%%%%%%%%%%%%

%%%%%%%%%%%%%%%%%%%%
\begin{frame}{A Banking Application}
  \begin{center}
    % \textsc{deposit \quad withdraw \quad query \quad interest}
    \fig{width = 0.60\textwidth}{figs/alice-bob}
    \red{Invariant: $\textsf{balance} \ge 0$}

    \vspace{0.50cm}
    Only \strongcolor{\textsc{withdraw}} on the same account are marked \strongcolor{strong}.

    \pause
    \vspace{0.30cm}
    One of {\strongcolor{\textsc{withdraw($60$)}}} aborts.
  \end{center}
\end{frame}
%%%%%%%%%%%%%%%%%%%%

%%%%%%%%%%%%%%%%%%%%
\begin{frame}{Related Work}
  \begin{center}
    \red{\large non-transactional}
  \end{center}

  \ncite{Attiya:SIAMJComput1998}

  \vspace{0.30cm}
  \ncite{NonSeq:SIAMJComput1998}

  \fig{width = 0.30\textwidth}{figs/multiple-consistency-levels}
\end{frame}
%%%%%%%%%%%%%%%%%%%%

%%%%%%%%%%%%%%%%%%%%
\begin{frame}{Related Work}
  \begin{center}
    \red{\large not fault-tolerant}
  \end{center}

  \ncite{BackEC:EuroSys2015}

  \vspace{0.30cm}
  \ncite{Li:OSDI2012}

  \vspace{0.30cm}
  \ncite{Li:ATC2018}
\end{frame}
%%%%%%%%%%%%%%%%%%%%

%%%%%%%%%%%%%%%%%%%%
\begin{frame}{Liveness of \unistore}
  \begin{center}
    \uncover<2->{How to ensure \red{liveness} despite data center failures?}
    \fig{width = 0.50\textwidth}{figs/design-challenges}

    \vspace{0.30cm}
    \uncover<3->{A transaction that is either \purple{strong}
      or \causalcolor{originates at a correct data center}
      eventually become \red{visible} at all correct data centers.}
  \end{center}
\end{frame}
% ~\footnote{某数据中心接收到一个事务,不代表这个事务对该数据中心上的客户``可见''。}
%%%%%%%%%%%%%%%%%%%%

%%%%%%%%%%%%%%%%%%%%
\begin{frame}{Liveness of Causal Transactions}
  \begin{center}
    $d_{1}$ crashes before $\causalcolor{t_{1}}$ is replicated to $d_{3}$.

    \vspace{0.20cm}
    \fig{width = 0.70\textwidth}{figs/liveness-causal}

    \pause
    Transaction \causalcolor{$t_{2}$} may never become visible at $d_{3}$.
  \end{center}
\end{frame}
%%%%%%%%%%%%%%%%%%%%

%%%%%%%%%%%%%%%%%%%%
\begin{frame}{Liveness of Causal Transactions}
  \begin{center}
    $d_{1}$ crashes before $\causalcolor{t_{1}}$ is replicated to $d_{3}$.

    \vspace{0.20cm}
    \fig{width = 0.70\textwidth}{figs/liveness-causal}

    \pause
    $d_{2}$ need to \red{forward} causal transactions to other data centers.
  \end{center}
\end{frame}
%%%%%%%%%%%%%%%%%%%%

%%%%%%%%%%%%%%%%%%%%
\begin{frame}{Liveness of Strong Transactions}
  \begin{center}
    $d_{1}$ crashes before $\causalcolor{t_{1}}$ is replicated to $d_{3}$.

    \vspace{0.50cm}
    \fig{width = 1.00\textwidth}{figs/liveness-strong}

    \pause
    \vspace{0.30cm}
    $\strongcolor{t_{2}}$ will never be visible at $d_{3}$. \\[3pt]
    \pause
    No transaction $\strongcolor{t_{3}}$ conflicting with $\strongcolor{t_{2}}$ can commit.
  \end{center}
\end{frame}
%%%%%%%%%%%%%%%%%%%%

%%%%%%%%%%%%%%%%%%%%
\begin{frame}{Liveness of Strong Transactions}
  \begin{center}
    A \strongcolor{strong} transaction should wait for all its
    \causalcolor{causal dependencies} to become \red{uniform} before committing. \\[3pt]
    {\footnotesize (eventually become visible at all correct data centers)}

    \vspace{0.30cm}
    \fig{width = 1.00\textwidth}{figs/liveness-strong}

    \pause
    $\causalcolor{t_{1}}$ will eventually be visible at $d_{3}$. \\[3pt]
    \pause
    $\strongcolor{t_{2}}$ will eventually be visible at $d_{3}$. \\[3pt]
    \pause
    $\strongcolor{t_{3}}$ may be committed at $d_{3}$. \\[3pt]
  \end{center}
\end{frame}
%%%%%%%%%%%%%%%%%%%%

%%%%%%%%%%%%%%%%%%%%
\begin{frame}{Minimizing the Latency of Strong Transactions}
  \begin{center}
    A \strongcolor{strong} transaction should wait for all its
    \causalcolor{causal dependencies} to become \red{uniform} before committing.

    \vspace{0.30cm}
    \fig{width = 1.00\textwidth}{figs/liveness-strong}

    \pause
    \vspace{0.20cm}
    \strongcolor{$t_{2}$} waits for \red{local} \causalcolor{$t_{1}$}
    to become uniform before committing.

    \pause
    \vspace{0.50cm}
    However, it may cost too much to wait for \red{remote} \causalcolor{causal dependencies}
    to become uniform.
  \end{center}
\end{frame}
%%%%%%%%%%%%%%%%%%%%

%%%%%%%%%%%%%%%%%%%%
\begin{frame}{Minimizing the Latency of Strong Transactions}
  \begin{center}
    Let \causalcolor{causal transactions} be executed on an (almost)
    \red{uniform snapshot} that may be slightly in the past.

    \vspace{0.50cm}
    \fig{width = 1.00\textwidth}{figs/liveness-strong}

    \pause
    \vspace{0.30cm}
    Make a \red{remote} \causalcolor{causal transaction} visible
    only \red{after it is uniform}.

    \pause
    \vspace{0.30cm}
    A \strongcolor{strong} transaction does \emph{not}
    need to wait for \red{remote} \causalcolor{causal dependencies}
    to become uniform before committing.
  \end{center}
\end{frame}
%%%%%%%%%%%%%%%%%%%%