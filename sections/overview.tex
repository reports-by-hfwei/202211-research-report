% overview.tex

%%%%%%%%%%%%%%%%%%%%
\begin{frame}{Background}
	\begin{center}
		Centralized Databases \emph{vs.} Distributed Databases

		\fig{width = 0.65\textwidth}{figs/Centralized-vs-Decentralized}
	\end{center}
\end{frame}
%%%%%%%%%%%%%%%%%%%%

%%%%%%%%%%%%%%%%%%%%
\begin{frame}{Data Consistency Problem}
	\begin{center}
		The Classic ``Data Partition + Data Replication'' Architecture
		\fig{width = 0.60\textwidth}{figs/partition-replication}

		\pause
		\red{(Distributed) Data Consistency Problem}
	\end{center}
\end{frame}
%%%%%%%%%%%%%%%%%%%%

%%%%%%%%%%%%%%%%%%%%
\begin{frame}{Data Consistency Problem}
	\begin{center}
		(Strong) \underline{\red{C}}onsistency, \underline{\red{A}}vailability,
		\underline{\red{L}}atency, \underline{\red{P}}artition tolerance
	\end{center}
	\begin{columns}[c]
		\column{0.60\textwidth}
		\begin{center}
			\fig{width = 0.50\textwidth}{figs/cap-theorem.png}
			CAP Theorem \\[2pt]
			\citepf{Brewer@PODC2000}
		\end{center}
		\column{0.40\textwidth}
		\begin{center}
			\fig{width = 0.95\textwidth}{figs/pacelc-tradeoff}

			\vspace{0.50cm}
			PACELC Tradeoff \\[2pt]
			\citepf{Abadi@Computer2012}
		\end{center}
	\end{columns}
\end{frame}
%%%%%%%%%%%%%%%%%%%%

%%%%%%%%%%%%%%%%%%%%
\begin{frame}{Data Consistency Problem}
	\begin{center}
		\fig{width = 0.40\textwidth}{figs/one-size-fit-all-text}

		\vspace{0.30cm}
		没有一劳永逸的解决方案
	\end{center}
\end{frame}
%%%%%%%%%%%%%%%%%%%%

%%%%%%%%%%%%%%%%%%%%
\begin{frame}{Consistency Models}
	\begin{center}
		Use various \blue{consistency models} to capture these tradeoffs
	\end{center}

	\begin{columns}[c]
		\column{0.50\textwidth}
		\begin{center}
			\fig{width = 0.90\textwidth}{figs/consistency-model-tree}
			\citepf{Bailis@VLDB2012}
		\end{center}
		\column{0.50\textwidth}
		\begin{center}
			\fig{width = 1.00\textwidth}{figs/consistency-models-csur2016}
			\citepf{Viotti@CSUR2016}
		\end{center}
	\end{columns}

	\pause
	\fig{width = 0.15\textwidth}{figs/strong-weak}
\end{frame}
%%%%%%%%%%%%%%%%%%%%

%%%%%%%%%%%%%%%%%%%%
\begin{frame}{My Researches}
	\begin{center}
		On the data consistency theory around consistency models:
	\end{center}

	\pause
	% \begin{columns}
	% 	\column{0.05\textwidth}
	% 	\column{0.90\textwidth}
	\begin{description}[<+->]
		\setlength{\itemsep}{8pt}
		\item[Computability:] What is possible or impossible?
		\item[Specification:] How to define consistency models?
		\item[Protocol:] How to design a ``good'' protocol? \\[3pt]
			How to prove its correctness?
		\item[Testing:] What is the complexity? \\[3pt] How to design efficient testing algorithms?
			% \item[协议验证:] TODO:
	\end{description}
	% 	\column{0.05\textwidth}
	% \end{columns}

	% \pause
	% \vspace{0.50cm}
	% \begin{center}
	% 	Classic problems with the ever-changing requirements
	% \end{center}
\end{frame}
%%%%%%%%%%%%%%%%%%%%

%%%%%%%%%%%%%%%%%%%%
\begin{frame}{My Researches (I)}
	\begin{center}
		% On read/write registers\footnote{读写寄存器,也就是读写变量,虽然最初与计算机系统中的``寄存器''概念相关,但已慢慢解耦。} ($\ge 2014$) \\[6pt]

		On non-transactional key-value stores \\[6pt]

		\fig{width = 0.60\textwidth}{figs/nosql-kv}
	\end{center}
\end{frame}
%%%%%%%%%%%%%%%%%%%%

%%%%%%%%%%%%%%%%%%%%
\begin{frame}{My Researches (I)}
	\ncite{Wei:TPDS2016}

	\vspace{0.30cm}
	\ncite{Wei:TC2017}

	\vspace{0.30cm}
	\ncite{Wei:PODC2020-Huang}
\end{frame}
%%%%%%%%%%%%%%%%%%%%

%%%%%%%%%%%%%%%%%%%%
\begin{frame}{My Researches (II)}
	\begin{center}
		On replicated data types\footnote{复制数据类型,是经典数据结构的分布式版本,如列表,集合,队列等。} ($\ge 2017$) \\[6pt]

		\begin{columns}[c]
			\column{0.50\textwidth}
			\fig{width = 0.55\textwidth}{figs/rdt-dbs}
			\column{0.50\textwidth}
			\fig{width = 0.65\textwidth}{figs/coeditor}
		\end{columns}
	\end{center}
\end{frame}
%%%%%%%%%%%%%%%%%%%%

%%%%%%%%%%%%%%%%%%%%
\begin{frame}{My Researches (II)}
	\ncite{Wei:PODC-BA2018}

	\vspace{0.30cm}
	\ncite{Wei:OPODIS2018}

	\vspace{0.30cm}
	\ncite{Wei:SRDS2020-Jiang}

	\vspace{0.30cm}
	\ncite{Wei:JOS2020-Ji}
\end{frame}
%%%%%%%%%%%%%%%%%%%%

%%%%%%%%%%%%%%%%%%%%
\begin{frame}{My Researches (III)}
	\begin{center}
		On distributed transactions\footnote{
			分布式事务。每个事务由一组操作构成,这些操作要么都成功,要么都失败。} ($\ge 2020$) \\[6pt]

		\fig{width = 0.50\textwidth}{figs/transactional-dbs}
	\end{center}
\end{frame}
%%%%%%%%%%%%%%%%%%%%

%%%%%%%%%%%%%%%%%%%%
\begin{frame}{My Researches (III)}
	\ncite{Wei:ATC2021}
\end{frame}
%%%%%%%%%%%%%%%%%%%%