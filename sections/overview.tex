% overview.tex

%%%%%%%%%%%%%%%%%%%%
\begin{frame}{Background}
	\begin{center}
		Centralized Databases \emph{vs.} Distributed Databases

		\fig{width = 0.65\textwidth}{figs/Centralized-vs-Decentralized}

		% \begin{columns}[c]
		% 	\column{0.50\textwidth}
		% 	% \fig{width = 0.75\textwidth}{figs/centralized}
		% 	\column{0.50\textwidth}
		% 	\fig{width = 0.75\textwidth}{figs/sina-weibo-world-map}
		% \end{columns}
	\end{center}
\end{frame}
%%%%%%%%%%%%%%%%%%%%

%%%%%%%%%%%%%%%%%%%%
\begin{frame}{Data Consistency Problem}
	\begin{center}
		``Data Partition + Data Replication''
		\fig{width = 0.60\textwidth}{figs/partition-replication}
		\red{Data Consistency Problem}
	\end{center}
\end{frame}
%%%%%%%%%%%%%%%%%%%%

%%%%%%%%%%%%%%%%%%%%
\begin{frame}{Data Consistency Problem}
	\begin{center}
		(Strong) \underline{\red{C}}onsistency, \underline{\red{A}}vailability,
		\underline{\red{L}}atency、\underline{\red{P}}atition tolerance
	\end{center}
	\begin{columns}[c]
		\column{0.60\textwidth}
		\begin{center}
			\fig{width = 0.50\textwidth}{figs/cap-theorem.png}
			{CAP Theorem}%~\ncite{}
		\end{center}
		\column{0.40\textwidth}
		\begin{center}
			\fig{width = 0.95\textwidth}{figs/pacelc-tradeoff}

			\vspace{0.50cm}
			{PACELC Tradeoff}
		\end{center}
	\end{columns}
\end{frame}
%%%%%%%%%%%%%%%%%%%%

%%%%%%%%%%%%%%%%%%%%
\begin{frame}{Consistency Models}
	\begin{center}
		Use \blue{consistency models} to capture these tradeoffs
	\end{center}

	\begin{columns}[c]
		\column{0.50\textwidth}
		\fig{width = 0.90\textwidth}{figs/consistency-model-tree}
		\column{0.50\textwidth}
		\fig{width = 1.00\textwidth}{figs/consistency-models-csur2016}
	\end{columns}
\end{frame}
%%%%%%%%%%%%%%%%%%%%

%%%%%%%%%%%%%%%%%%%%
\begin{frame}{My Researches}
	\begin{center}
		Researches on data consistency around consistency models:
	\end{center}

	\begin{columns}
		\column{0.10\textwidth}
		\column{0.80\textwidth}
		\begin{description}[Computability]
			\setlength{\itemsep}{8pt}
			\item[Computability:] What is possible or impossible?
			\item[Protocol:] How to design fast, scalable, and fault-tolerant protocols?
			\item[Testing:] What is the complexity? \\[3pt] How to design efficient testing algorihthms?
				% \item[协议验证:] TODO:
		\end{description}
		\column{0.10\textwidth}
	\end{columns}

	\pause
	\vspace{0.60cm}
	\begin{center}
		Classic problems with the ever-changing requirements
	\end{center}
\end{frame}
%%%%%%%%%%%%%%%%%%%%

%%%%%%%%%%%%%%%%%%%%
\begin{frame}{My Researches}
	\begin{center}
		Research I ($\ge 2012$): Read/Write Register\footnote{读写寄存器,也就是读写变量,虽然最初与计算机系统中的``寄存器''概念相关,但已慢慢解耦。} \\[6pt]

		\fig{width = 0.60\textwidth}{figs/dsss}
		Distributed NoSQL Key-Value Stores (TODO: 重新画图)

		TODO: +research outcomes
	\end{center}
\end{frame}
%%%%%%%%%%%%%%%%%%%%

%%%%%%%%%%%%%%%%%%%%
\begin{frame}{My Researches}
	\begin{center}
		Research II ($\ge 2017$): Replicated Data Types\footnote{复制数据类型,是经典数据类型的分布式版本,如列表,集合,队列等。} \\[6pt]

		\begin{columns}[c]
			\column{0.50\textwidth}
			\fig{width = 0.60\textwidth}{figs/rdt-dbs}
			\column{0.50\textwidth}
			\fig{width = 0.65\textwidth}{figs/coeditor}
		\end{columns}

		TODO: +research outcomes
	\end{center}
\end{frame}
%%%%%%%%%%%%%%%%%%%%

%%%%%%%%%%%%%%%%%%%%
\begin{frame}{My Researches}
	\begin{center}
		Research III ($\ge 2020$): Distributed Transactions\footnote{
			分布式事务。每个事务由一组操作构成,这些操作要么全成功,要么全不成功。}

		TODO: +research outcomes
		TODO: +logos
	\end{center}
\end{frame}
%%%%%%%%%%%%%%%%%%%%