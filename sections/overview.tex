% overview.tex

%%%%%%%%%%%%%%%%%%%%
\begin{frame}{}
	\begin{center}
		集中式数据库 \emph{vs.} 分布式数据库

		\fig{width = 0.65\textwidth}{figs/Centralized-vs-Decentralized}

		% \begin{columns}[c]
		% 	\column{0.50\textwidth}
		% 	% \fig{width = 0.75\textwidth}{figs/centralized}
		% 	\column{0.50\textwidth}
		% 	\fig{width = 0.75\textwidth}{figs/sina-weibo-world-map}
		% \end{columns}
	\end{center}
\end{frame}
%%%%%%%%%%%%%%%%%%%%

%%%%%%%%%%%%%%%%%%%%
\begin{frame}{}
	\begin{center}
		``数据分区 + 数据副本'' 系统架构
		\fig{width = 0.60\textwidth}{figs/partition-replication}
		\red{数据一致性问题}
	\end{center}
\end{frame}
%%%%%%%%%%%%%%%%%%%%

%%%%%%%%%%%%%%%%%%%%
\begin{frame}{}
	\begin{center}
		强一致性、高可用性、延迟、分区容忍性四者之间的权衡关系
	\end{center}
	\begin{columns}[c]
		\column{0.60\textwidth}
		\begin{center}
			\fig{width = 0.50\textwidth}{figs/cap-theorem.png}
			{CAP Theorem}%~\ncite{}
		\end{center}
		\column{0.40\textwidth}
		\begin{center}
			\fig{width = 0.95\textwidth}{figs/pacelc-tradeoff}

			\vspace{0.50cm}
			{PACELC Tradeoff\\ \ncite{abadi:computer12}}
		\end{center}
	\end{columns}
\end{frame}
%%%%%%%%%%%%%%%%%%%%

%%%%%%%%%%%%%%%%%%%%
\begin{frame}{}
	\begin{center}
		使用强弱不同的数据一致性模型刻画不同的权衡
	\end{center}

	\begin{columns}[c]
		\column{0.50\textwidth}
		\fig{width = 0.90\textwidth}{figs/consistency-model-tree}
		\column{0.50\textwidth}
		\fig{width = 1.00\textwidth}{figs/consistency-models-csur2016}
	\end{columns}
\end{frame}
%%%%%%%%%%%%%%%%%%%%

%%%%%%%%%%%%%%%%%%%%
\begin{frame}{}
	\begin{center}
		以数据一致性模型为核心的数据一致性理论
	\end{center}

	\begin{columns}
		\column{0.10\textwidth}
		\column{0.80\textwidth}
		\begin{description}
			\setlength{\itemsep}{8pt}
			\item[理论基础:] 什么是不可能的? 下界是什么?
			\item[协议设计:] 如何设计容错、高可扩展、高性能的协议?
			\item[系统测试:] 复杂度是多少? 如何设计高效的测试算法?
				% \item[协议验证:] TODO:
		\end{description}
		\column{0.10\textwidth}
	\end{columns}

	\pause
	\vspace{0.60cm}
	\begin{center}
		经典的分布式计算理论在技术发展的需求导向下与时俱进
	\end{center}
\end{frame}
%%%%%%%%%%%%%%%%%%%%

%%%%%%%%%%%%%%%%%%%%
\begin{frame}{}
	\begin{center}
		研究工作第一阶段 ($\ge 2012$):读写寄存器 (Read/Write Register) \\[6pt]

		\fig{width = 0.60\textwidth}{figs/dsss}
		分布式 NoSQL Key-Value 数据库 (TODO: 重新画图)

		TODO: +research outcomes
	\end{center}
\end{frame}
%%%%%%%%%%%%%%%%%%%%

%%%%%%%%%%%%%%%%%%%%
\begin{frame}{}
	\begin{center}
		研究工作第二阶段 ($\ge 2017$):复制数据类型 (Replicated Data Types) \\[6pt]

		\begin{columns}[c]
			\column{0.50\textwidth}
			\fig{width = 0.80\textwidth}{figs/coeditor}
			\column{0.50\textwidth}
			% \fig{width = 0.80\textwidth}{figs/rdt-dbs}
			TODO: Jupiter/Redis/Riak
		\end{columns}

		TODO: +research outcomes
	\end{center}
\end{frame}
%%%%%%%%%%%%%%%%%%%%

%%%%%%%%%%%%%%%%%%%%
\begin{frame}{}
	\begin{center}
		研究工作第二阶段 ($\ge 2020$):分布式事务 (Distributed Transactions)

		TODO: +research outcomes
		TODO: +logos
	\end{center}
\end{frame}
%%%%%%%%%%%%%%%%%%%%